\documentclass[prl,amsmath,twocolumn,amssymb,superscriptaddress,linenumbers]{revtex4-1}
%\usepackage[T1]{fontenc}
\usepackage{times}
%\usepackage[latin9]{inputenc}
%\setcounter{secnumdepth}{3}
\usepackage{graphicx}
\usepackage{amsmath, braket}
\usepackage{amssymb}
\usepackage{natbib}
%\usepackage{titling}
%\usepackage{babel}
%\usepackage{bm, ulem} % seemed to be putting underlines in a few undesirable situations
\usepackage[normalem]{ulem} % enables strikeout (\sout) without changing \emph
\usepackage[usenames]{xcolor}
%\usepackage{wrapfig}
\usepackage{mhchem}
\usepackage[parse-numbers=false]{siunitx} % handles numbers with units nicely in both math mode and text mode
\usepackage{xspace} % correctly handles spaces after macros
% \usepackage{multibbl} % enables two bibliographies (for SI)
\usepackage[colorlinks]{hyperref}

% separate bibliography stuff
% \newbibliography{main} % multibbl option
% \newbibliography{appendix} % multibbl option

\newcommand{\tas}{\ce{TaS2}\xspace}
\newcommand{\nbse}{\ce{NbSe2}\xspace}
\newcommand{\hBN}{h-\ce{BN}\xspace}

\newcommand{\Kv}{\ensuremath{\mathbf{K}}\xspace}
\newcommand{\Kpv}{\ensuremath{\mathbf{K'}}\xspace}
\newcommand{\K}{K\xspace}
\newcommand{\Kp}{K$^\prime$\xspace}

\newcommand{\Rn}{\ensuremath{R_n}}
\newcommand{\Rgr}{\ensuremath{R_\mathrm{graphite}}}
\newcommand{\muB}{\ensuremath{\mu_\mathrm{B}}}
\newcommand{\kB}{\ensuremath{k_\mathrm{B}}}
\newcommand{\kF}{\ensuremath{k_\mathrm{F}}}
\newcommand{\kFv}{\ensuremath{\mathbf{k}_\mathrm{F}}}

\newcommand{\Tc}{\ensuremath{T_c}}
\newcommand{\Tcz}{\ensuremath{T_{c0}}}

\newcommand{\Pauli}{p}%{\mathrm{P}}
\newcommand{\Hp}{\ensuremath{H_\Pauli}}
\newcommand{\Hso}{\ensuremath{H_\mathrm{so}}}
\newcommand{\Hperp}{\ensuremath{H_\perp}}
\newcommand{\Hpara}{\ensuremath{H_\parallel}}
\newcommand{\Hc}{\ensuremath{H_{c2}^\parallel}}

\newcommand{\Bpara}{\ensuremath{B_\parallel}}
\newcommand{\Bso}{\ensuremath{B_\mathrm{so}}}
\newcommand{\Bsov}{\ensuremath{\mathbf{B}_\mathrm{so}}}

\newcommand{\Dso}{\ensuremath{\Delta_\mathrm{so}}}
\newcommand{\Dsok}{\ensuremath{\Delta_\mathrm{so}(\mathbf{k})}}
\newcommand{\Dvb}{\ensuremath{\Delta_\mathrm{vb}}}
\newcommand{\Dvbk}{\ensuremath{\Delta_\mathrm{vb}(\mathbf{k})}}
\newcommand{\lso}{\ensuremath{\lambda_\mathrm{so}}}
\newcommand{\tso}{\ensuremath{\tau_\mathrm{so}}}
\newcommand{\alphaR}{\ensuremath{\alpha_\mathrm{R}}}

\newcommand{\revise}[1]{\noindent\colorbox{lightgray}{#1}\xspace}
\newcommand{\new}[1]{#1\xspace}
\newcommand{\cut}[1]{\xspace}
% \newcommand{\new}[1]{{\color{blue}#1}\xspace}
% \newcommand{\cut}[1]{{\color{red}\sout{#1}}\xspace}

%%%%%%%%%%%%%%%%%%%%%%
\begin{document}

\title{Precision Oriented Kinetic Extraction (P.O.K.E.) : A last resort method for lithographic liftoff}

\author{Michael~R.~Sinko} \email{msinko@andrew.cmu.edu}
\author{Devashish~P.~Gopalan}% \email{dgopalan@andrew.cmu.edu}
\affiliation{Department of Physics, Carnegie Mellon University, Pittsburgh, PA 15213}
\author{Dacen~Waters}% \email{}
\affiliation{Department of Physics, Carnegie Mellon University, Pittsburgh, PA 15213}

\author{Benjamin~M.~Hunt} \email{bmhunt@andrew.cmu.edu}
\affiliation{Department of Physics, Carnegie Mellon University, Pittsburgh, PA 15213}

\begin{abstract}
 
\end{abstract}
\maketitle

Lithographic patterning of metallic leads is a common nanofabrication technique used in a wide array of applications. One persistent issue is in this process is difficulty in the removal of excess evaporated metal around the features of the mask. This problem can be mitigated to a great extent by careful preparation of the sample before lithographic patterning by the choice of mask/resist materials/layers, better design of lithographic features, and liftoff process. In cases where these steps have not worked, or have not been taken, the most common step to complete the liftoff of undesired regions is to ultrasonicate the sample in a container of acetone. This method is viable for samples that are not sensitive to such vibrations, but would be inadviseable for other samples, such as ones that contain layered 2D materials or stacks of 2D materials forming van der Waals (vdW) heterostructures.  
The vdW forces that adhere these types of samples to the substrate, approx $0.35J/m^2$ according to DFT calculations (other estimates have ranged from $0.09J/m^2$ to $0.9J/m^2$) are significantly less than the energy densities a system in a sonication bath is subject to (what are those energies? source?). This would imply that it is quite likely for a 2D material adhered to a substrate through vdW forces to be knocked loose by the energy input by a sonication bath, nonetheless the power of a sonication probe which can be 1000x greater than that of a bath.
We present a method for removing loosely flagging metal, peninsulas, isthmuses, and trapped islands of metal resulting from a poor liftoff. Our technique, as shown in (Fig.~\ref{fig-schematic}a), uses a fine needle mounted to a micromanipulator with $~$1 $\mu$m control. The specific mounting method is used for a micromanipulator adapted to holding glass slides in a standard 2D material dry transfer setup. A fine needle (tip diameter=2um???) (the unplugging needle for a wirebonder was used in the original version: I should try to find better needles!!!) is inserted in reverse (to preserve the fine tip) through the corner of a small block at an angle between 30 and 45 degrees below horizontal(1cm x 1cm x 4mm?) of PDMS (poly di-methylsilicane?) that is adhered along the edge of standard glass slide.  The glass slide is then mounted to a micromanipulator, while the substrate remains in an acetone bath in a shallow walled dish (we used a watchglass) to be placed under a long working distance lens of a microscope. After the sample area with poor liftoff is located in the objective area, the micromanipulator is used to bring the needle into conatact with the substrate away from sensitive written features. At this point, a few strategies can be used which will be discussed later. This technique has been used with success to remove flags from between features with sub-micron separations.

The role of the PDMS in the mounting of the needle to the micromanipulator is twofold. First, it allows for the angle of the needle relative to the substrate to be changed by inserting the back of the needle into the PDMS block at a different angle; secondly, the PDMS applies a spring force to the needle as it is pushed down slightly onto the substrate. The tip of the needle should glide smoothly forward after it is lowered and touches the surface, the PDMS. Caution should be used when retracting the needle, as the tip will pull backwards along the substrate the same distance it was pushed forward by pushing down on the substrate with the PDMS as the tip was lowered. The needle's fine tip allows for easy movement along the sides of large features (10+ $\mu$m) that are separated by a similar distance, in these instances, the needle can be pushed along the edge of the large feature, with the length of the needle parallel to the edge of the feature, this will serve to shear the flagging metal off of the feature's side. This can be repeated as necessary to remo ve flags of metal from large features by reorienting the needle or the substrate to parallel each edge. Trapped islands of metal that are fully enclosed or nearly fully enclosed with only a thin isthmus connecting it to a bulk flag should be approached differently. Regions with diameters larger than the diameter of the needle's tip should be accesible, with larger regions being easier than smaller regions.  Here, the tip of the needle can be brought down near the middle of the region and dragged around slightly until a tear at the edge of the enclosing feature can be formed and exploited. Many times, the small flag of metal will stick to the tip of the needle after it is removed from its connected feature. typically, this can be due to part of the flag being held down by the needle, lifting the needle from the surface of the substrate can allow the flag to float away. 



 















\end{document}
